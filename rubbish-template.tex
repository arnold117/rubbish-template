\documentclass[10pt]{article}

%% ========== Packages ==========
\usepackage[utf8]{inputenc}
\usepackage[T1]{fontenc}
\usepackage{geometry}
\usepackage{fancyhdr}
\usepackage{titlesec}
\usepackage{hyperref}
\usepackage{graphicx}
\usepackage{amsmath,amssymb}
\usepackage{xcolor}
\usepackage{tcolorbox}
\usepackage{multicol}
\usepackage{enumitem}
\usepackage{float}
\usepackage{caption}
\usepackage{indentfirst}
\usepackage{fontspec}  % Requires XeLaTeX or LuaLaTeX

\tcbuselibrary{skins,breakable}

%% ========== Fonts ==========
\setmainfont{Times New Roman}
\newfontfamily\baskervillefont{Baskerville}[
  ItalicFont=Baskerville-Italic,
  BoldFont=Baskerville-Bold,
]
\newfontfamily\georgiafont{Georgia}

%% ========== Page Geometry ==========
\geometry{
  a4paper,
  left=1.5cm,
  right=1.5cm,
  top=2.0cm,
  bottom=2.0cm,
  headheight=15pt
}

%% ========== Colors (exact from original: #007146) ==========
\definecolor{rubbishgreen}{HTML}{007146}
\definecolor{abstractbg}{RGB}{230,243,235}
\definecolor{doiboxbg}{HTML}{007146}
\definecolor{linkblue}{HTML}{000080}

%% ========== Hyperlink Setup ==========
\hypersetup{
  colorlinks=true,
  linkcolor=linkblue,
  citecolor=linkblue,
  urlcolor=linkblue
}

%% ========== Header and Footer ==========
\pagestyle{fancy}
\fancyhf{}
\fancyfoot[L]{%
  {\footnotesize\color{rubbishgreen}\textit{Rubbish}\,}%
  {\footnotesize| Volume XX | February 2026 | XXXX}%
}
\fancyfoot[R]{\footnotesize\textbf{PAGE \thepage}}
\renewcommand{\headrulewidth}{0pt}
\renewcommand{\footrulewidth}{0pt}

\fancypagestyle{firstpage}{%
  \fancyhf{}
  \fancyfoot[L]{\footnotesize\textbf{PAGE \thepage}}
  \fancyfoot[R]{%
    {\footnotesize\color{rubbishgreen}\textit{Rubbish}\,}%
    {\footnotesize| Volume XX | February 2026 | XXXX}%
  }
  \renewcommand{\headrulewidth}{0pt}
  \renewcommand{\footrulewidth}{0pt}
}

%% ========== Section Formatting (all bold italic, green) ==========
\titleformat{\section}
  {\normalfont\fontsize{12}{14}\selectfont\bfseries\itshape\color{rubbishgreen}}{\thesection}{0.5em}{}
\titlespacing{\section}{0pt}{12pt plus 2pt minus 1pt}{4pt plus 1pt minus 1pt}

\titleformat{\subsection}
  {\normalfont\fontsize{10.5}{13}\selectfont\bfseries\itshape\color{rubbishgreen}}{\thesubsection}{0.5em}{}
\titlespacing{\subsection}{0pt}{8pt plus 2pt minus 1pt}{3pt plus 1pt minus 1pt}

\titleformat{\subsubsection}
  {\normalfont\fontsize{9}{11}\selectfont\bfseries\itshape\color{rubbishgreen}}{\thesubsubsection}{0.5em}{}
\titlespacing{\subsubsection}{0pt}{6pt plus 1pt minus 1pt}{2pt plus 1pt minus 1pt}

%% ========== Body text indent ==========
\setlength{\parindent}{21pt}

%% ========== Article Metadata Commands ==========
\newcommand{\journalname}{RUBBISH}
\newcommand{\journalsubtitle}{JOURNAL OF WASTE \& CIRCULAR ECONOMY}
\newcommand{\journalvolume}{Vol. 7 | No. 1 | SPRING 2024}
\newcommand{\articletype}{Research Article}
\newcommand{\accesstype}{Open Access}
\newcommand{\received}{Received 00th January 20xx,}
\newcommand{\accepted}{Accepted 00th January 20xx}
\newcommand{\articledoi}{DOI: xx.xxxx/xxxx}
\newcommand{\articletitle}{Template for Rubbish Submissions Rubbish Submissions Rubbish Submission}

%% ========== Document Begin ==========
\begin{document}
\thispagestyle{firstpage}

%% ============================================================
%% TITLE BLOCK (single column, full width)
%% ============================================================

%% ---------- Top Header: Logo + Info Box ----------
\noindent
\begin{minipage}[t]{0.55\textwidth}
  \vspace{0pt}
  \includegraphics[width=3.5cm]{logo.png}
\end{minipage}%
\hfill
\begin{minipage}[t]{0.40\textwidth}
  \vspace{0pt}
  \begin{tcolorbox}[
    colback=abstractbg,
    colframe=rubbishgreen,
    boxrule=0pt,
    leftrule=3pt,
    arc=0pt,
    left=10pt, right=10pt, top=8pt, bottom=8pt,
    fontupper=\small
  ]
    \textbf{\accesstype}\\[2pt]
    \textbf{\articletype}\\[4pt]
    \received\\
    \accepted\\[4pt]
    {\color{rubbishgreen}\textbf{\articledoi}}
  \end{tcolorbox}
\end{minipage}

\vspace{14pt}

%% ---------- Title ----------
{\noindent\fontsize{48}{52}\selectfont\bfseries\color{rubbishgreen}\journalname\par}

\vspace{6pt}

{\noindent\fontsize{18}{22}\selectfont\bfseries \articletitle\par}

\vspace{8pt}

%% ---------- Authors ----------
{\noindent\fontsize{10.5}{13}\selectfont
  First Author\textsuperscript{1,*,$\dagger$},
  Second Author\textsuperscript{2,*,$\dagger$},
  Third Author\textsuperscript{2}
  and Fourth Author\textsuperscript{2,*}
\par}

\vspace{3pt}

{\noindent\fontsize{9}{11}\selectfont
  \textsuperscript{1}First Institution
  and \textsuperscript{2}Second Institution
\par}

\vspace{5pt}

{\noindent\georgiafont\fontsize{9}{11}\selectfont
  \textsuperscript{*}\,\href{mailto:abc@uni.edu}{\underline{abc@uni.edu}};
  \href{mailto:alphbeta@lab.edu}{\underline{alphbeta@lab.edu}}
\par}

\vspace{2pt}

{\noindent\fontsize{9}{11}\selectfont
  \textsuperscript{$\dagger$}Contributed equally.
\par}

\vspace{10pt}

%% ---------- Abstract Box ----------
\begin{tcolorbox}[
  colback=abstractbg,
  colframe=rubbishgreen,
  boxrule=0pt,
  leftrule=3pt,
  arc=0pt,
  left=10pt, right=10pt, top=8pt, bottom=8pt,
  fontupper=\fontsize{10}{12}\selectfont
]
  The abstract of the manuscript should cover a presentation of the interest
  or relevance of these data for the broader community; a very brief preview
  of the data type(s) produced, the methods used, and information relevant to
  data validation. As well as the potential uses of these data and implications
  for the field. Please minimize the use of abbreviations and do not cite
  references in the abstract. As this article type is focussed on describing
  a dataset, conclusions or interpretive insights are not required.
\end{tcolorbox}

\vspace{6pt}

%% ---------- Keywords ----------
{\noindent\fontsize{10.5}{13}\selectfont
  \textbf{Keywords:} Classification~1; Classifications~2; Classifications~3\par}

\vspace{8pt}

%% ============================================================
%% MAIN BODY (two-column layout using multicol)
%% ============================================================
\begin{multicols}{2}
\baskervillefont

\section{Introduction}

This is the epigraph text, should you like to add one. This is the epigraph
text, should you like to add one. This is the epigraph text, should you like to
add one. This is the epigraph text, should you like to add one.

\section{Methodology}

This is the epigraph text, should you like to add one. This is the epigraph
text, should you like to add one. This is the epigraph text, should you like to
add one. This is the epigraph text, should you like to add one.

\subsection{Title of Methodology}

This is the epigraph text, should you like to add one. This is the epigraph
text, should you like to add one. This is the epigraph text, should you like to
add one. This is the epigraph text, should you like to add one.

\subsubsection{Title of Methodology}

This is the epigraph text, should you like to add one. This is the epigraph
text, should you like to add one. This is the epigraph text, should you like to
add one. This is the epigraph text, should you like to add one.

\subsubsection{Title of Methodology}

This is the epigraph text, should you like to add one. This is the epigraph
text, should you like to add one. This is the epigraph text, should you like to
add one. This is the epigraph text, should you like to add one.

\section{Results and Discussion}

\subsection{Title}

This is the epigraph text, should you like to add one.

\subsubsection{Title}

\subsubsection{Title}

This is the epigraph text, should you like to add one. This is the epigraph
text, should you like to add one. This is the epigraph text, should you like to
add one. This is the epigraph text, should you like to add one.

\section{Conclusion}

This is the epigraph text, should you like to add one. This is the epigraph
text, should you like to add one. This is the epigraph text, should you like to
add one. This is the epigraph text, should you like to add one.

%% ========== Acknowledgements ==========
\section*{Acknowledge}

This is the epigraph text, should you like to add one. This is the epigraph
text, should you like to add one. This is the epigraph text, should you like to
add one. This is the epigraph text, should you like to add one.

%% ========== References ==========
\section*{Reference}

\noindent [1] This is the epigraph text, should you like to add one. This is
the epigraph text, should you like to add one. This is the epigraph text,
should you like to add one. This is the epigraph text, should you like to add
one.

\medskip

\noindent [2] This is the epigraph text, should you like to add one. This is
the epigraph text, should you like to add one. This is the epigraph text,
should you like to add one. This is the epigraph text, should you like to add
one.

\end{multicols}

\end{document}
