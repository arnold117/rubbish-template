\documentclass{rubbish}

%% ================================================================
%%  ARTICLE METADATA
%%  Fill in your paper's information below.
%% ================================================================

\rubbishtitle{Your Paper Title Goes Here: Capitalise Major Words and Keep It Under Two Lines If Possible}
\rubbishmid{MID: xx.xxxx/xxxx}                   % Manuscript ID assigned by editorial board
\rubbishreceived{Received 00th January 20xx,}     % Submission date
\rubbishaccepted{Accepted 00th January 20xx}      % Acceptance date
\rubbishtype{Research Article}                     % Research Article / Review Article / etc.

%% --- Authors ---
%%   Superscript numbers link to affiliations.
%%   * = corresponding author, † = equal contribution.
\rubbishauthors{
  First Author\textsuperscript{1,*,$\dagger$},
  Second Author\textsuperscript{2,$\dagger$},
  Third Author\textsuperscript{2}
  and Fourth Author\textsuperscript{2,*}
}

%% --- Affiliations (appear below author names) ---
\rubbishaffiliations{
  \textsuperscript{1}First Institution
  and \textsuperscript{2}Second Institution\\[1pt]
  \textsuperscript{*}\,\georgiafont
    \href{mailto:abc@uni.edu}{\underline{abc@uni.edu}};
    \href{mailto:alphbeta@lab.edu}{\underline{alphbeta@lab.edu}}\\[1pt]
  \textsuperscript{$\dagger$}Contributed equally.
}

%% --- Keywords ---
\rubbishkeywords{keyword one; keyword two; keyword three; keyword four}

%% ================================================================
\begin{document}

\makerubbishtitle

\begin{rubbishabstract}
  Write your abstract here. The abstract should briefly present: (1)~the
  motivation or relevance of your work; (2)~a summary of the methods used;
  (3)~key results or findings; and (4)~the main conclusions. Avoid citations,
  abbreviations, and mathematical notation. Keep it concise---250 words or
  fewer is recommended.
\end{rubbishabstract}

%% ================================================================
%%  BODY — Two-column layout starts here.
%%  Replace the guide content below with your own paper.
%% ================================================================
\begin{rubbishbody}

\section{How to Use This Template}

Welcome to the official \LaTeX\ template for \textit{Rubbish}. This
document doubles as a user guide: when compiled, it demonstrates every
formatting feature the template supports. Replace this content with your
own manuscript before submission.

\subsection{Compilation}

This template must be compiled with \textbf{XeLaTeX} (not pdfLaTeX),
because \texttt{rubbish.cls} uses the \texttt{fontspec} and
\texttt{xeCJK} packages for font selection and CJK character support.
A typical compilation sequence:

\begin{enumerate}[leftmargin=*, itemsep=2pt]
  \item \texttt{xelatex rubbish-template.tex}
  \item \texttt{xelatex rubbish-template.tex} \quad (run twice for
        cross-references)
\end{enumerate}

\noindent On Overleaf, set the compiler to \textbf{XeLaTeX} via
Menu $\to$ Compiler $\to$ XeLaTeX.

\subsection{Required Files}

Place the following files in the same directory:

\begin{itemize}[leftmargin=*, itemsep=2pt]
  \item \texttt{rubbish.cls} --- the document class (do not modify
        unless you know what you are doing)
  \item \texttt{rubbish-template.tex} --- your paper (this file)
  \item \texttt{logo.png} --- the \textit{Rubbish} journal logo
  \item Any figures referenced by your paper
\end{itemize}

\section{Metadata Setup}

All article metadata is configured in the preamble of this
\texttt{.tex} file via simple commands. The most important ones:

\begin{itemize}[leftmargin=*, itemsep=2pt]
  \item \texttt{\textbackslash rubbishtitle\{...\}} --- your paper
        title
  \item \texttt{\textbackslash rubbishmid\{MID: ...\}} --- manuscript
        ID assigned by the editorial board
  \item \texttt{\textbackslash rubbishreceived\{...\}} and
        \texttt{\textbackslash rubbishaccepted\{...\}} --- dates
  \item \texttt{\textbackslash rubbishtype\{...\}} --- article type
        (Research Article, Review, etc.)
\end{itemize}

\subsection{Authors and Affiliations}

Authors are set with \texttt{\textbackslash rubbishauthors\{...\}}.
Use superscript numbers to link authors to affiliations, \texttt{*} for
corresponding authors, and \texttt{\$\textbackslash dagger\$} for equal
contribution:

\begin{verbatim}
\rubbishauthors{
  Alice\textsuperscript{1,*},
  Bob\textsuperscript{2}
}
\end{verbatim}

\noindent Affiliations are set with
\texttt{\textbackslash rubbishaffiliations\{...\}} and appear directly
below the author names. Include institutions, contact emails, and
contribution notes in that block.

\subsection{Abstract and Keywords}

The abstract is written inside the \texttt{rubbishabstract} environment.
Keywords are set separately with
\texttt{\textbackslash rubbishkeywords\{...\}} and rendered
automatically below the abstract box.

\section{Text Formatting}

\subsection{Section Hierarchy}

The template provides three levels of headings, all rendered in
\textbf{\textit{\color{rubbishgreen}bold italic green}}:

\begin{enumerate}[leftmargin=*, itemsep=2pt]
  \item \texttt{\textbackslash section\{...\}} --- 12\,pt
  \item \texttt{\textbackslash subsection\{...\}} --- 10.5\,pt
  \item \texttt{\textbackslash subsubsection\{...\}} --- 9\,pt
\end{enumerate}

\noindent Sections are numbered automatically. For unnumbered headings
(e.g.\ Acknowledgements), use starred variants:
\texttt{\textbackslash section*\{...\}}.

\subsection{Inline Styles}

\begin{itemize}[leftmargin=*, itemsep=2pt]
  \item \textbf{Bold}: \texttt{\textbackslash textbf\{...\}}
  \item \textit{Italic}: \texttt{\textbackslash textit\{...\}}
  \item \underline{Underline}: \texttt{\textbackslash underline\{...\}}
  \item {\color{rubbishgreen}Coloured}:
        \texttt{\{\textbackslash color\{rubbishgreen\}...\}}
  \item \texttt{Monospace}: \texttt{\textbackslash texttt\{...\}}
\end{itemize}

\subsection{Lists}

Numbered lists use \texttt{enumerate}; bullet lists use
\texttt{itemize}. Add \texttt{[leftmargin=*]} to remove the default
left indent, as done throughout this guide.

\subsection{Footnotes}

Add footnotes with \texttt{\textbackslash footnote\{...\}}. They
are automatically numbered and placed at the bottom of the
current column\footnote{Like this one.}. In the two-column layout,
footnotes appear at the bottom of the column where they are
called, not at the bottom of the page.

\subsection{Hyperlinks}

The template uses the \texttt{hyperref} package. Available commands:

\begin{itemize}[leftmargin=*, itemsep=2pt]
  \item \texttt{\textbackslash href\{url\}\{text\}} --- clickable
        text, e.g.\
        \href{https://github.com/arnold117/rubbish-template}{template
        repository}
  \item \texttt{\textbackslash url\{...\}} --- displays the full URL,
        e.g.\ \url{https://example.com}
  \item \texttt{\textbackslash href\{mailto:...\}\{...\}} --- email
        links, already used in the affiliations block
\end{itemize}

\noindent All links are rendered in dark blue
(\texttt{linkblue}, \texttt{\#000080}) as defined in
\texttt{rubbish.cls}.

\subsection{CJK Support}

Chinese and Japanese text is supported natively via
\texttt{xeCJK}. Example: 这是中文测试,日本語テスト。

The default CJK font is PingFang SC (macOS). For other platforms,
edit \texttt{rubbish.cls} and change
\texttt{\textbackslash setCJKmainfont}: use SimSun on Windows or
Noto Serif CJK SC on Linux.

\subsection{Code Blocks}

For short inline code, use \texttt{\textbackslash texttt\{...\}}
to produce \texttt{monospaced text}. For multi-line code blocks,
use the \texttt{lstlisting} environment, which renders inside a
green-accented box with syntax highlighting:

\begin{lstlisting}[language=Python]
import olefile
ole = olefile.OleFileIO("Rubbish.doc")
data = ole.openstream("Data").read()
# PNG at byte offset 583, 62717 bytes
\end{lstlisting}

\noindent Specify the language with \texttt{[language=...]}:

\begin{lstlisting}[language=bash]
xelatex rubbish-template.tex
xelatex rubbish-template.tex
\end{lstlisting}

\noindent The code box style (green border, light green background,
bold green keywords) is defined in \texttt{rubbish.cls} via
\texttt{\textbackslash lstset}. Supported languages include Python,
bash, C, Java, and many more.

\section{Figures, Tables, and Equations}

\subsection{Figures}

Within the two-column body, wrap figures in a
\texttt{minipage\{\textbackslash columnwidth\}} with
\texttt{\textbackslash captionof} for the caption. Here is the
pattern used in actual \textit{Rubbish} submissions:

\begin{verbatim}
\noindent
\begin{minipage}{\columnwidth}
\centering
\includegraphics[width=
  \columnwidth]{your-fig.png}
\captionof{figure}{Your caption.}
\label{fig:example}
\end{minipage}
\end{verbatim}

\noindent And here is a live example using the journal logo:

\vspace{6pt}
\noindent
\begin{minipage}{\columnwidth}
\centering
\includegraphics[width=0.5\columnwidth]{logo.png}
\captionof{figure}{The \textit{Rubbish} journal logo, included here
as a demonstration of figure insertion.}
\label{fig:logo}
\end{minipage}
\vspace{6pt}

\noindent Reference figures in text with
\texttt{Figure\textasciitilde\textbackslash ref\{fig:logo\}},
which produces: Figure~\ref{fig:logo}.

\subsection{Tables}

Tables use \texttt{booktabs} for clean horizontal rules. Wrap in a
\texttt{minipage} to keep within one column:

\vspace{6pt}
\noindent
\begin{minipage}{\columnwidth}
\captionof{table}{An Example Table}
\vspace{2pt}
\centering
\small
\begin{tabular}{@{}lcc@{}}
  \toprule
  \textbf{Item} & \textbf{Value} & \textbf{Unit} \\
  \midrule
  Length         & 42             & cm             \\
  Width          & 17             & cm             \\
  Quality        & ---            & Rubbish        \\
  \bottomrule
\end{tabular}
\end{minipage}
\vspace{6pt}

\noindent Use \texttt{\textbackslash toprule},
\texttt{\textbackslash midrule}, and \texttt{\textbackslash bottomrule}
instead of \texttt{\textbackslash hline}. Use
\texttt{\textbackslash captionof\{table\}\{...\}} for the caption.

\subsection{Equations}

Inline math: $E = mc^2$. Display equations:

\begin{equation}
  \int_{-\infty}^{\infty} e^{-x^2}\,dx = \sqrt{\pi}
  \label{eq:gaussian}
\end{equation}

\noindent Reference: Equation~\ref{eq:gaussian}. Multi-line:

\begin{align}
  a &= b + c \\
  d &= e \times f
\end{align}

\noindent Use \texttt{equation*} or
\texttt{\textbackslash[...\textbackslash]} for unnumbered equations.

\subsection{Full-Width Elements}

To insert a figure or table that spans \textbf{both columns}, you
must temporarily exit and re-enter the two-column environment:

\begin{verbatim}
\end{rubbishbody}

\noindent
\begin{minipage}{\textwidth}
\centering
\includegraphics[width=0.8\textwidth]
  {wide-figure.png}
\captionof{figure}{A full-width figure.}
\end{minipage}

\begin{rubbishbody}
\end{verbatim}

\noindent This breaks out of \texttt{multicols}, places the
full-width element, and restarts the two-column layout. The same
pattern works for wide tables. Note that a column break will occur
at the exit point, so place full-width elements at natural section
boundaries when possible.

\section{Citations and References}

References in \textit{Rubbish} are formatted \textbf{manually} at the
end of the document---no BibTeX or Biber. Cite in text as \texttt{[1]},
\texttt{[2]}, etc. In the \textbf{Reference} section, list entries with
\texttt{\textbackslash noindent [n]} and separate with
\texttt{\textbackslash medskip}. Order by first appearance in the text.

\section{Conclusion}

Replace this guide with your own content. Keep the preamble metadata
commands and the \texttt{rubbishabstract} / \texttt{rubbishbody}
structure intact. Good luck---remember, the acceptance rate is
100\%.

%% ========== Acknowledgements ==========
%%   Optional. Thank funding agencies, collaborators, etc.
\section*{Acknowledgements}

The authors thank the \textit{Rubbish} editorial board for their
unwavering commitment to accepting everything.

%% ========== Conflicts of Interest ==========
%%   Required. Declare any conflicts or state "None."
\section*{Conflicts of Interest}

The authors declare no conflicts of interest.

%% ========== Funding ==========
%%   Required. List grants/funding sources or state "None."
\section*{Funding}

This research received no external funding.

%% ========== Ethics Statement ==========
%%   Required if human/animal subjects are involved.
%%   Remove this section if not applicable.
\section*{Ethics Statement}

Not applicable.

%% ========== References ==========
\section*{References}

\noindent [1] Author, A.B.; Author, C.D. Title of the Article.
\textit{Journal Name} \textbf{Year}, \textit{Volume}, page--page.

\medskip

\noindent [2] Author, E.F. \textit{Title of the Book}; Publisher:
City, Country, Year.

\medskip

\noindent [3] Author, G.H. Title of the Conference Paper. In
\textit{Proceedings of the Conference Name}, City, Country,
Date; page--page.

\end{rubbishbody}

\end{document}
